
%%%%%%%%%%%%%%%%%%%%%%% file typeinst.tex %%%%%%%%%%%%%%%%%%%%%%%%%
%
% This is the LaTeX source for the instructions to authors using
% the LaTeX document class 'llncs.cls' for contributions to
% the Lecture Notes in Computer Sciences series.
% http://www.springer.com/lncs       Springer Heidelberg 2006/05/04
%
% It may be used as a template for your own input - copy it
% to a new file with a new name and use it as the basis
% for your article.
%
% NB: the document class 'llncs' has its own and detailed documentation, see
% ftp://ftp.springer.de/data/pubftp/pub/tex/latex/llncs/latex2e/llncsdoc.pdf
%
%%%%%%%%%%%%%%%%%%%%%%%%%%%%%%%%%%%%%%%%%%%%%%%%%%%%%%%%%%%%%%%%%%%


\documentclass[runningheads,a4paper]{llncs}

\usepackage{amssymb}
\setcounter{tocdepth}{3}
\usepackage{graphicx}
\usepackage[utf8]{inputenc}

\usepackage{url}
\urldef{\mailsa}\path|{rodrigo.g.branco,dmbpaiva,istela}@gmail.com|    
\newcommand{\keywords}[1]{\par\addvspace\baselineskip
\noindent\keywordname\enspace\ignorespaces#1}

\usepackage{epstopdf}

\begin{document}

\mainmatter  % start of an individual contribution

% first the title is needed
\title{AccTrace: Acessibilidade nas fases de Engenharia de Requisitos, Projeto e Codificação de Software}

% a short form should be given in case it is too long for the running head

%VERIFICAR
\titlerunning{AccTrace}

% the name(s) of the author(s) follow(s) next
%
% NB: Chinese authors should write their first names(s) in front of
% their surnames. This ensures that the names appear correctly in
% the running heads and the author index.
%
\author{Rodrigo Gonçalves de Branco%
%\thanks{Please note that the LNCS Editorial assumes that all authors have used
%the western naming convention, with given names preceding surnames. This determines
%the structure of the names in the running heads and the author index.}%
\and Débora Maria Barroso Paiva \and Maria Istela Cagnin}

%VERIFICAR
\authorrunning{AccTrace}
% (feature abused for this document to repeat the title also on left hand pages)

% the affiliations are given next; don't give your e-mail address
% unless you accept that it will be published
\institute{Faculdade de Computação - FACOM\\
Universidade Federal de Mato Grosso do Sul - UFMS\\
Cidade Universitária, Campo Grande - MS, Brasil\\
\mailsa\\
\url{http://facom.ufms.br/}}

%
% NB: a more complex sample for affiliations and the mapping to the
% corresponding authors can be found in the file "llncs.dem"
% (search for the string "\mainmatter" where a contribution starts).
% "llncs.dem" accompanies the document class "llncs.cls".
%

\toctitle{Lecture Notes in Computer Science}
\tocauthor{Authors' Instructions}
\maketitle


\begin{abstract}
The abstract should summarize the contents of the paper and should
contain at least 70 and at most 150 words. It should be written using the
\emph{abstract} environment.
\keywords{Accessibility, Requirement Traceability, Software Development
Process, CASE Tool}
%\keywords{We would like to encourage you to list your keywords within
%the abstract section}
\end{abstract}


\section{Introdução}

A Internet se consolidou como um dos principais meios de comunicação, disseminação da informação e fornecimento de serviços nos dias atuais, sendo que determinados tipos de informações são disponibilizados apenas através deste meio. Fornecer o acesso a estas informações para uma imensa variedade de dispositivos, com tecnologias e especificações diferentes, se tornou um desafio. E, principalmente, pessoas com necessidades especiais, devido a suas características inerentes, necessitam de produtos especialmente desenvolvidos para este fim.

Construir produtos acessíveis não é uma tarefa fácil e este assunto é alvo de muitas pesquisas \cite{lazar:04,brajnik:06,zeng:05}, existindo propostas específicas para integrar usabilidade e acessibilidade aos processos de Engenharia de Software \cite{springerlink:10.1007/978-3-642-02713-0,maia:10}. Contudo, devido a fatores como a recente exposição do tema nos meios de comunicação e a deficiência no treinamento e formação  dos desenvolvedores, muitos deles sequer sabem como codificar para tornar seus produtos acessíveis \cite{1630123,alves:11}.

A utilização de ferramentas que apóiam os profissionais na construção de boas soluções acessíveis é muito comum, pois estas aumentam a produtividade e diminuem o esforço necessário. Estas ferramentas podem variar entre ferramentas CASE, IDEs, frameworks, simuladores, validadores, avaliadores e até mesmo uma mistura destas citadas. Contudo, pesquisas mostram que as ferramentas disponíveis deixam a desejar no quesito de apoio à acessibilidade \cite{Trewin:2010:ACT:1805986.1806029}.

A situação é agravada pela constatação de que, apesar de existirem vários estudos sobre rastreabilidade de requisitos genéricos \cite{5970169,292398,5485417,6405269}, poucos estudos têm indicado como ocorre a evolução dos requisitos de acessibilidade durante o processo de desenvolvimento das aplicações web \cite{analuizadias:2010}, e como os desenvolvedores utilizarão essa informação para construir o produto.

Este artigo apresenta o AccTrace, uma ferramenta CASE desenvolvida como um
plugin do Eclipse para, através da rastreabilidade dos requisitos de
acessibilidade, entregar ao desenvolvedor informações relevantes para a
construção de um produto acessível. O AccTrace foi baseado em um processo de
desenvolvimento de softwares que incluía tarefas de acessibilidades baseado na
ISO 12207, chamado MTA \cite{maia:10}. Durante o processo de desenvolvimento do
software, as referências entre requisitos e modelos UML, juntamente com a nova
abordagem de especificar quais serão as técnicas de implementação de
acessibilidade para estes relacionamentos, resultam em um comentário
personalizado no código fonte, que é recuperado em tempo real detalhando as
informações descritas anteriormente.

\subsection{Exemplo Motivacional}

A rastreabilidade dos requisitos geralmente é feita utilizando matrizes de
rastreabilidade \cite{guo:2009:OBI:1681515.1682933}. Contudo, nas abordagens tradicionais, os
requisitos de acessibilidade não são diferenciados dos outros requisitos. Desta
forma, a implementação de tais requisitos não se torna explícita.
Desenvolvedores experientes podem não encontrar problemas na implementação, mas
desenvolvedores iniciantes podem não ter o mesmo sucesso.

É certo que o documento WCAG 2.0 pode e deve ser usado como documento de
referência, pois ele contém critérios de sucesso, técnicas para implementação,
etc. Contudo, este documento não é de fácil leitura para iniciantes,
principalmente no que tange encontrar a informação necessária. Por este motivo,
inclusive, que normalmente o documento é usado para avaliar e validar produtos
prontos.

O desenvolvedor seria beneficiado caso a informação de o que deve ser feito, no
que tange acessibilidade, estivesse disponível. Isto é exatamente o que nossa
abordagem tenta resolver.

\section{Trabalhos Relacionados}

O tema Acessibilidade no Processo de desenvolvimento de software ainda
é pouco explorado, com poucos trabalhos publicados a respeito
\cite{maia:10,5599835}. Em contrapartida, podem ser encontrados diversos
estudos que dizem respeito ao rastreamento de requisitos
\cite{5970169,292398,5485417,6405269}, bem como encontrar estudos que utilizam
ontologias para o mapeamento do domínio no processo de desenvolvimento ou na
rastreabilidade dos requisitos \cite{5223183,6511842,4148940,5362244}. Contudo,
não foi encontrado na literatura estudos específicos sobre a rastreabilidade de
requisitos de acessibilidade durante o processo de desenvolvimento de software.



\begin{thebibliography}{4}

\bibitem{lazar:04} Lazar, J., Dudley-Sponaugle, A.,  Greenidge, K.-D.: Improving web accessibility:
a study of webmaster perceptions. Computers in Human Behavior, 20(2), 269--288. (2004)

\bibitem{brajnik:06} Brajnik, G.: Web Accessibility Testing: When the Method Is the Culprit. In: Miesenberger, K., Klaus, J., Zagler, W. L., e Karshmer, A. I. (eds.) ICCHP 2006, vol. 4061, pp. 156--163. Springer (2006)

\bibitem{zeng:05} Parmanto, B. e Zeng, X.: Metric for Web accessibility evaluation. JASIST, 56(13), 1394--1404. (2005)

\bibitem{springerlink:10.1007/978-3-642-02713-0} Moreno, L., Martínez, P., Ruiz-Mezcua, B.: Integrating HCI in a Web Accessibility Engineering Approach. In: Stephanidis, C., (ed.) Universal Access in Human-Computer Interaction. Applications and Services, vol. 5616, pp. 745--754. Springer (2009)

\bibitem{maia:10} Maia, L. S.: Um processo para o desenvolvimento de aplicações Web Acessíveis.
Master Thesis, UFMS (2010)

\bibitem{1630123} Kavcic, A.: Software Accessibility: Recommendations and Guidelines. In: Computer as a Tool. EUROCON, The International Conference, vol. 2, pp 1024--1027. (2005)

\bibitem{alves:11} Alves, D. D.: Acessibilidade no Desenvolvimento de Software Livre. Master Thesis, UFMS (2011)

\bibitem{Trewin:2010:ACT:1805986.1806029} Trewin, S., Cragun, B., Swart, C., Brezin, J., Richards, J.: Accessibility challenges
and tool features: an IBM Web developer perspective. In: Proceedings of the
2010 International Cross Disciplinary Conference on Web Accessibility. W4A 2010, pp 32:1--32:10. ACM. (2010)

\bibitem{analuizadias:2010} Dias, A. L., de Mattos Fortes, R. P., Masiero, P. C., Goularte, R.: Uma Revisão
Sistemática sobre a inserção de Acessibilidade nas fases de desenvolvimento da Engenharia
de Software em sistemas Web. In: Proceedings of the IX Symposium on Human
Factors in Computing Systems. IHC 2010, pp. 39--48. SBC (2010)

\bibitem{5970169} Ali, N., Gueheneuc, Y., Antoniol, G.: Trust-Based Requirements Traceability.
In Program Comprehension. ICPC 2011, pp. 111--120. IEEE (2011)

\bibitem{292398} Gotel, O. C. Z., Finkelstein, A. C. W.: An analysis of the requirements traceability
problem. In: Requirements Engineering, Proceedings of the First International
Conference on, pp. 94--101 (1994)

\bibitem{5485417} Soonsongtanee, S., Limpiyakorn, Y.: Enhancement of requirements traceability
with state diagrams. In: Computer Engineering and Technology (ICCET), 2nd International Conference on, vol. 2, pp. 248--252 (2010)

\bibitem{6405269} Mader, P., Egyed, A.: Assessing the effect of requirements traceability for software
maintenance. In: Software Maintenance (ICSM), 28th IEEE International Conference on, pp. 171--180 (2012)

\bibitem{guo:2009:OBI:1681515.1682933} guo, Y., Yang, M., Wang, J., Yang, P., Li, F.: An Ontology
Based Improved Software Requirement Traceability Matrix. In: Proceedings of the
2009 Second International Symposium on Knowledge Acquisition and Modeling,
Vol. 01, pp. 160--163. IEEE (2009)

\bibitem{5599835} Moulin, C., Sbodio, M.L.: Improving the accessibility and
efficiency of e-Government processes. In: Cognitive Informatics (ICCI), 2010 9th
IEEE International Conference on, pp. 603--610 (2010)

\bibitem{5223183} Assawamekin, N., Sunetnanta, T., Pluempitiwiriyawej, C.: MUPRET: An
Ontology-Driven Traceability Tool for Multiperspective Requirements Artifacts.
In: Computer and Information Science, 2009. ICIS 2009. Eighth IEEE/ACIS
International Conference on, pp. 943--948 (2009)

\bibitem{6511842} Martins, J., Machado, R.: Ontologies for Product and Process
Traceability at Manufacturing Organizations: A Software Requirements Approach.
In: Quality of Information and Communications Technology (QUATIC), 2012 Eighth
International Conference on the, pp. 353--358 (2012)

\bibitem{4148940} Noll, R., Ribeiro, M.: Ontological Traceability over
the Unified Process. In: Engineering of Computer-Based Systems, 2007. ECBS '07.
14th Annual IEEE International Conference and Workshops on the, pp.
249--255 (2007)

\bibitem{5362244} guo, Y., Yang, M., Wang, J., Yang, P., Li, F.: An Ontology Based Improved
Software Requirement Traceability Matrix. In: Knowledge Acquisition and
Modeling, 2009. KAM '09. Second International Symposium on, vol. 1, pp.
160--163 (2009)


\end{thebibliography}

\end{document}
